%%%%%%%%%%%%%%%%%%%%%%%%%%%%%%%%%%%%%%%%%
% Beamer Presentation
% LaTeX Template
% Version 1.0 (10/11/12)
%
% This template has been downloaded from:
% http://www.LaTeXTemplates.com
%
% License:
% CC BY-NC-SA 3.0 (http://creativecommons.org/licenses/by-nc-sa/3.0/)
%
%%%%%%%%%%%%%%%%%%%%%%%%%%%%%%%%%%%%%%%%%

%----------------------------------------------------------------------------------------
%	PACKAGES AND THEMES
%----------------------------------------------------------------------------------------

\documentclass{beamer}

\mode<presentation> {

% The Beamer class comes with a number of default slide themes
% which change the colors and layouts of slides. Below this is a list
% of all the themes, uncomment each in turn to see what they look like.

%\usetheme{default}
%\usetheme{AnnArbor}
%\usetheme{Antibes}
%\usetheme{Bergen}
%\usetheme{Berkeley}
%\usetheme{Berlin}
%\usetheme{Boadilla}
%\usetheme{CambridgeUS}
%\usetheme{Copenhagen}
%\usetheme{Darmstadt}
%\usetheme{Dresden}
%\usetheme{Frankfurt}
%\usetheme{Goettingen}
%\usetheme{Hannover}
%\usetheme{Ilmenau}
%\usetheme{JuanLesPins}
%\usetheme{Luebeck}
\usetheme{Madrid}
%\usetheme{Malmoe}
%\usetheme{Marburg}
%\usetheme{Montpellier}
%\usetheme{PaloAlto}
%\usetheme{Pittsburgh}
%\usetheme{Rochester}
%\usetheme{Singapore}
%\usetheme{Szeged}
%\usetheme{Warsaw}

% As well as themes, the Beamer class has a number of color themes
% for any slide theme. Uncomment each of these in turn to see how it
% changes the colors of your current slide theme.

%\usecolortheme{albatross}
%\usecolortheme{beaver}
%\usecolortheme{beetle}
%\usecolortheme{crane}
%\usecolortheme{dolphin}
%\usecolortheme{dove}
%\usecolortheme{fly}
%\usecolortheme{lily}
%\usecolortheme{orchid}
%\usecolortheme{rose}
%\usecolortheme{seagull}
%\usecolortheme{seahorse}
%\usecolortheme{whale}
%\usecolortheme{wolverine}

%\setbeamertemplate{footline} % To remove the footer line in all slides uncomment this line
%\setbeamertemplate{footline}[page number] % To replace the footer line in all slides with a simple slide count uncomment this line

%\setbeamertemplate{navigation symbols}{} % To remove the navigation symbols from the bottom of all slides uncomment this line
}

\usepackage{graphicx} % Allows including images
\usepackage{booktabs} % Allows the use of \toprule, \midrule and \bottomrule in tables
\usepackage{subcaption}

%----------------------------------------------------------------------------------------
%	TITLE PAGE
%----------------------------------------------------------------------------------------

\title[Impulzne nevronske mreže]{Spodbujevano učenje na impulznih nevronskih mrežah} % The short title appears at the bottom of every slide, the full title is only on the title page

\author{Matjaž Pogačnik} % Your name
\institute[UL] % Your institution as it will appear on the bottom of every slide, may be shorthand to save space
{
Univerza v Ljubljani \\ % Your institution for the title page
Fakulteta za računalništvo in informatiko \\
\medskip
\textit{mp24170@student.uni-lj.si} \\% Your email address
\hfill \\
Mentor: prof. dr. Zoran Bosnić
}
\date{\today} % Date, can be changed to a custom date

\begin{document}

\begin{frame}
\titlepage % Print the title page as the first slide
\end{frame}

\begin{frame}
\frametitle{Overview} % Table of contents slide, comment this block out to remove it
\tableofcontents % Throughout your presentation, if you choose to use \section{} and \subsection{} commands, these will automatically be printed on this slide as an overview of your presentation
\end{frame}

%----------------------------------------------------------------------------------------
%	PRESENTATION SLIDES
%----------------------------------------------------------------------------------------

%------------------------------------------------

\section{Uvod} % A subsection can be created just before a set of slides with a common theme to further break down your presentation into chunks

\begin{frame}
\frametitle{Uvod}

\begin{itemize}
    \item Impulzne nevronske mreže SNN združujejo čas, energijsko učinkovitost in biološko realističnost.
    \item Informacija je kodirana v zaporedju in času impulzov.
    \item Pri ANN čas zanemarjen ali obravnavan v diskretnih korakih.
    \item Učenje preko lokalnih pravil namesto gradientov.
\end{itemize}
\vspace{0.3cm}
\begin{itemize}
    \item \textbf{Problem:} kako izvajati spodbujevano učenje na impulznih nevronskih mrežah.
    \item \textbf{Cilj:} razviti biološko smiselno arhitekturo, ki omogoča učenje tudi pri oddaljenih nagradah.
\end{itemize}
\begin{figure}
\includegraphics[width=1.0\linewidth]{figs/pre-sin-post.png}
\end{figure}
\end{frame}

\section{R-STDP model}

\begin{frame}
\frametitle{Lokalno pravilo za učenje}
\begin{itemize}
    \item \textbf{Problem:} katere povezave so bile odgovorne za proženje izhoda? (problem pripisovanja odgovornosti)
    \vspace{0.3cm}

    \item Sinaptična plastičnost odvisna od časovne razporeditve impulzov (STDP).
    \item Sledi upravičenosti (eligibility traces)

        \vspace{0.3cm}
    \item \textbf{Zakaj STDP?} 
    \begin{itemize}
        \item biološko smiselno,
        \item ne potrebuje gradientov,
        \item deluje naravno s časom impulzov,
    \end{itemize}

\end{itemize}
\begin{figure}
\includegraphics[width=0.5\linewidth]{figs/STDP.png}\\
{\tiny Source: \cite{safa}}  
\end{figure}
\end{frame}

%------------------------------------------------

\begin{frame}
\frametitle{Nevromodulirana STDP}
\begin{itemize}
    \item STDP ne vključuje informacije o nagradi. Razširimo z dopaminsko modulacijo.
    \item Dopamin globalno modulira plastičnost sinaps (R-STDP).
    \item Sinapse si zapomnijo preteklo aktivnost (sled upravičenosti).
        \vspace{0.3cm}
\item \textbf{Problem:} če nagrada pride prepozno, R-STDP ne ve več, katera odločitev je bila prava.
\end{itemize}
%Pri nevromodulirani STDP dodamo globalni signal nagrade, ki ga predstavlja dopamin.
%Sinapse ne posodabljamo takoj, ampak si zapomnijo, katera aktivnost je bila pomembna – temu pravimo sled upravičenosti.
%Ko pride nagrada, se ojačajo samo te sinapse.

\begin{figure}
\includegraphics[width=0.5\linewidth]{figs/RSTDP.png}
\end{figure}
\end{frame}

%------------------------------------------------
\subsection{Igra Pong}
\begin{frame}
\frametitle{Igra Pong}
\begin{itemize}
    \item Stanje - presinaptični nevron.
    \item Akcija - postsinaptični nevron.
    \item Nagrada - koncentracija dopamina.
    \item Gradient aproksimiramo preko sinaps z visoko sledjo upravičenosti. 
\end{itemize}
\begin{figure}
  \centering
  \begin{minipage}{0.3\textwidth}
    \centering
    \includegraphics[width=\textwidth]{figs/pong_setup.png}\\
    {\tiny Source: \cite{pong}}
  \end{minipage}
  \hfill
  \begin{minipage}{0.68\textwidth}
    \centering
    \includegraphics[width=\textwidth]{figs/mean_reward.png}
  \end{minipage}
\end{figure}

\end{frame}

%------------------------------------------------

\subsection{Mrežni svet}

\begin{frame}
\frametitle{Mrežni svet}
\begin{itemize}
    \item Verjetnost izbire akcije $a$ v stanju $i$ $\pi(a | i)$ - utež med vhodnim nevronom (stanje) in izhodnim (akcija).
    \item Rezultat učenja je zvišana sinaptična utež za pravilno akcijo.
    \item \textbf{Problem zakasnjene nagrade.}
\end{itemize}
\begin{figure}
\includegraphics[width=0.5\linewidth]{figs/best-rstdp.png}
\end{figure}
\end{frame}

%------------------------------------------------

\section{TD-učenje in model akter-kritik}
\begin{frame}
\frametitle{TD-učenje in model akter-kritik}
\begin{itemize}
    \item TD omogoča postopno propagacijo nagrade nazaj skozi stanja,
    \item Kritik ocenjuje pričakovano nagrado stanja,
    \item uporabimo zakasnjene in vzbujajoče/inhibitorne povezave.
    \item Akter izbira akcije,
    \item biološka povezava: bazalni gangliji + dopamin.
\end{itemize}
\begin{columns}[c] % [c] = vertical center alignment

\begin{column}{0.45\textwidth}
\[
    V(s_t) \leftarrow V(s_t) + \alpha \, \delta_t
\]
\[
    \delta_t = r_{t+1} + \gamma V(s_{t+1}) - V(s_t)
\]
\end{column}

\begin{column}{0.55\textwidth}
\centering
\includegraphics[width=\linewidth]{figs/actor-critic.png}
\end{column}

\end{columns}

\end{frame}

%------------------------------------------------

\subsection{Rezultati}

\begin{frame}
\frametitle{Rezultati}
\begin{itemize}
    \item Rezultat učenja je zvišana sinaptična utež za pravilno akcijo in višanje pričakovane nagrade skozi čas.
\end{itemize}
\begin{figure}
  \centering
  \begin{minipage}{0.39\textwidth}
    \centering
    \includegraphics[width=\textwidth]{figs/best.png}
  \end{minipage}
  \hfill
  \begin{minipage}{0.6\textwidth}
    \centering
    \includegraphics[width=\textwidth]{figs/progress.png}
  \end{minipage}
\end{figure}
\end{frame}

%------------------------------------------------


\section{Zaključek}
\begin{frame}
\frametitle{Zaključek}
\begin{itemize}
    \item Uspešna izvedba spodbujevanega učenja na impulznih nevronskih mrežah.
    \item Model deluje, vendar je občutljiv na hiperparametre.
    \item Negativne nagrade niso bile uporabljene (nadaljnje delo: \textit{Izognitveno obnašanje}).
    \item Vzvratne povezave niso bile uporabljene (nadaljnje delo: \textit{Rekurenčne povezave}).
\end{itemize}
\end{frame}

%------------------------------------------------

%------------------------------------------------

\begin{frame}
\frametitle{Reference}
\footnotesize{
\begin{thebibliography}{99}


\bibitem[Safa, 2024]{safa}
A. Safa (2024).
\newblock Continual Learning in Bio-plausible Spiking Neural Networks with Hebbian and Spike Timing Dependent Plasticity: A Survey and Perspective.
\newblock \emph{arXiv preprint}.
\newblock \url{https://doi.org/10.48550/arXiv.2407.17305}

\bibitem[IBM, 2023]{ibmrl}
IBM Think Blog (2023).
\newblock Reinforcement Learning.
\newblock \url{https://www.ibm.com/think/topics/reinforcement-learning}

\bibitem[Wunderlich et~al., 2019]{pong}
T. Wunderlich in~sod. (2019).
\newblock Demonstrating Advantages of Neuromorphic Computation: A Pilot Study.
\newblock \emph{Frontiers in Neuroscience}, 13:260.
\newblock \url{https://doi.org/10.3389/fnins.2019.00260}


\end{thebibliography}
}
\end{frame}



%------------------------------------------------

%----------------------------------------------------------------------------------------

\end{document} 

